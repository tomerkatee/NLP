%% LyX 2.3.8 created this file.  For more info, see http://www.lyx.org/.
%% Do not edit unless you really know what you are doing.
\documentclass[english]{article}
\usepackage[T1]{fontenc}
\usepackage[latin9]{inputenc}
\usepackage{geometry}
\geometry{verbose,tmargin=1in,bmargin=1in,lmargin=1in,rmargin=1in}
\usepackage{textcomp}
\usepackage{amssymb}
\usepackage{graphicx}
\usepackage{setspace}
\usepackage{babel}
\begin{document}
\begin{center}
{\LARGE{}Gradient Descent Very Brief Review}{\LARGE\par}
\par\end{center}

\begin{center}
{\large{}Eivi Katz \quad{} Tomer Katee}{\large\par}
\par\end{center}

\begin{center}
{\large{}214208704 \qquad{}214166027}{\large\par}
\par\end{center}

\begin{center}
{\large{}May 31, 2024}{\large\par}
\par\end{center}

\begin{center}
\textbf{Abstract}
\par\end{center}

Gradient descent is an iterative optimization algorithm for finding
the local minimum of a function. It and its variants are with no doubt
the most common optimization algorithms which are used in ML.

\textbf{\Large{}\begin{flushleft}1}{\large{}\quad{}}\textbf{\Large{}Introduction\end{flushleft}}{\Large\par}

\noindent Optimization refers to the task of minimizing/maximizing
an objective function $f(x)$ parameterized by $x$. In machine/deep
learning terminology, it\textquoteright s the task of minimizing the
cost/loss function $\mathcal{L}(\theta)$ parameterized by the model\textquoteright s
parameters $\theta\in R^{d}$. This objective function might be represented
in many ways - it could be a simple linear function such that $\theta$
is just one matrix, as well as a 175B parameters transformer architecture,
such that $\theta$ is a collection of large number of matrices and
bias terms. Optimization algorithms (in the case of minimization)
have one of the following goals:
\begin{itemize}
\item Find the global minimum of the objective function. This is feasible
if the objective function is convex, i.e. any local minimum is a global
minimum.
\item Find the lowest possible value of the objective function within its
neighborhood. That\textquoteright s usually the case if the objective
function is not convex as the case in most deep learning problems.
\end{itemize}
\textbf{\Large{}\begin{flushleft}2}{\large{}\quad{}}\textbf{\Large{}Algorithm\end{flushleft}}{\Large\par}

\noindent Let\textquoteright s denote our objective function by $\mathcal{L}$
(it is often referred to it as the loss function), and let\textquoteright s
say it is parameterized with some parameters which we will denote
with $\theta$. Additionally, let\textquoteright s denote an additional
hyperparameter the algorithm uses, called the learning rate, with
$\eta$. The Vanilla Gradient Descent Algorithm suggests following
these following steps:
\begin{enumerate}
\item Start by Randomly initialize values for $\theta_{0}$
\item Update $\theta$ values: $\theta_{t+1}$ = $\theta_{t}$ \textminus{}
$\eta\nabla L(\theta)$
\item Repeat until the slope is approximately flat. Namely, $\frac{\partial\mathcal{L}(\theta)}{\partial\theta}\thickapprox0$
\end{enumerate}
\begin{onehalfspace}
\textbf{\large{}2.1}{\large{}\quad{}}\textbf{\large{}Learning Rate}{\large\par}
\end{onehalfspace}

\begin{onehalfspace}
\noindent How big the steps gradient descent takes into the direction
of the local minimum are determined by the learning rate, which figures
out how fast or slow we will move towards the optimal weights. To
illustrate the effects of a wrong/right choice of the learning rate,
Figure 1 is attached.
\end{onehalfspace}

\begin{center}
\includegraphics[viewport=0bp 0bp 749.698bp 562.847bp,scale=0.75]{pasted1}
\par\end{center}
\end{document}
